\hypertarget{index_intro_sec}{}\section{Introducción}\label{index_intro_sec}
El kit de velocista\-Edu es un kit desarrollado por Vicent Colomar y Andreu Marsal para falicitar el acceso a los componentes necesarios para montar un robot siguelineas a cualquier persona interesada en construir su primer robot siguelineas competitivo. El kit incluye un shield para una placa Arduino Nano con el driver de motores, dos leds y un buzzer y los pines necesarios para conectar 4 sensores de infrarrojos, un pin de encoder por motor y los pines de I2\-C y puerto serie expuestos. El kit tambien incluye la placa para montar los sensores infrarrojos que detectarán la linea del suelo y las placas para añadir el encoder a los motores.

Con esta configuración se pretende que el robot sea de un facil montaje y sencillo de programar, pero que tenga la posibilidad de hacer desarrollos más complejos segun se va avanzando en el aprendizaje de la robótica.

La librería que se propone es solo una pequeña ayuda para que los primeros pasos con el kit sean lo más sencillos posibles.

En la pagina oficial de la liga nacional de robotica de competición (\href{http://lnrc.es/}{\tt http\-://lnrc.\-es/}) se puede encontrar más información sobre los eventos que se desarrollan a lo largo de la geografía española tanto para estudiantes como para gente con conocimientos más avanzados en los que se ponen a prueba los robots construidos y programados por los concursasntes. En la sección de estudiantes se puede encontrar más información sobre los kits disponibles con vídeos sobre su montaje.\hypertarget{index_install_sec}{}\section{Instalación}\label{index_install_sec}
Copiar la carpeta completa del proyecto en la carpeta de librerías que crea Arduino. Si el espacio en el disco donde se tiene que copiar la librería es muy escaso se pueden eliminar las carpetas \char`\"{}examples\char`\"{}, \char`\"{}doc\char`\"{} y \char`\"{}.\-git\char`\"{}\hypertarget{index_examples}{}\section{Ejemplos}\label{index_examples}
La librería incluye unos ejemplos de uso sencillos. Una vez instalada la librería los ejemplos deberían aparecer en la sección de ejemplos del I\-D\-E Arduino Los ejemplos disponibles ahora mismo son\-:
\begin{DoxyItemize}
\item lectura\-Boton\-: Muestra un mensaje por el puerto serie cada vez que se suelta el botón de la placa principal del kit
\item lectura\-Encoders\-: Muestra por el puerto serie la cuenta de los encoders. Según se muevan las ruedas la cuenta debería incrementarse. Siempre contará hacia arriba, ya que el encoder en cuadratura no está disponible
\item lectura\-Leds\-: Muestra por el puerto serie la lectura analógica de los sensores infrarrojos
\item parpadeo\-Leds\-: Hace que los leds de la placa del kit parpadeen. Solo uno deberia estar encendido en cada momento
\item siguelineas\-: Implementación de un control P\-D para seguir una linea negra sobre un fondo blanco. Los paramentros del P\-D no están optimizados 
\end{DoxyItemize}